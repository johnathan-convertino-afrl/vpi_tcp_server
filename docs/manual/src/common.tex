\begin{titlepage}
  \begin{center}

  {\Huge VPI\_TCP\_SERVER}

  \vspace{25mm}

  \includegraphics[width=0.90\textwidth,height=\textheight,keepaspectratio]{img/AFRL.png}

  \vspace{25mm}

  \today

  \vspace{15mm}

  {\Large Jay Convertino}

  \end{center}
\end{titlepage}

\tableofcontents

\newpage

\section{Usage}

\subsection{Introduction}

\par
VPI TCP Server is a library which allows for a Verilog simulation to interface with a TCP port.
This library provides three functions.
\begin{itemize}
\item setup\_tcp\_server(ADDRESS, PORT), RETURNS File Descriptor (FD)
\item recv\_tcp\_server(PORT, VECTOR), RETURNS number of bytes received (non-blocking, 0 is nothing available)
\item send\_tcp\_server(PORT, VECTOR), RETURNS number of bytes send (non-blocking, 0 is nothing written)
\end{itemize}

\par
Library supports up to 256 TCP server instances. Each instance is setup by
setup\_tcp\_server. This returns a descriptor for that instance. Then that descriptor
is used for nothing. The field PORT is used to associate setup\_tcp\_server with a recv\_tcp\_server
and a send\_tcp\_server. This can be done in multiple calls.

You can use the following for including the library in your project:

\begin{lstlisting}[language=XML]
dep_vpi:
  depend:
    - AFRL:vpi:tcp_server:1.0.0

targets:
  default: &default
    description: Default file set.
    filesets: [src, dep, dep_vpi]
\end{lstlisting}

\subsection{Dependencies}

\par
The following are the dependencies of the cores.

\begin{itemize}
  \item fusesoc 2.X
  \item iverilog (simulation)
\end{itemize}

\subsubsection{fusesoc\_info Depenecies}
\begin{itemize}
\item dep\_tb
	\begin{itemize}
	\item AFRL:utility:sim\_helper
	\end{itemize}
\item dep\_gen
	\begin{itemize}
	\item AFRL:utility:generators:1.0.0
	\end{itemize}
\end{itemize}


\section{Architecture}
\par
This VPI library provides three functions for the user to use during simulation for creating a TCP server. They are setup\_tcp\_server, recv\_tcp\_server, and send\_tcp\_server.
These are used to setup the server on a port, receive data, and send data respectivly. These functions use ringbuffers and multithreading to seperate server I/O from the simulation
so TCP access will not slow down the simulation.
\par
The setup\_tcp\_server is given an address, usually local or a ethernet port address, to use and a port. It will attempt to use this information and create an active connection.
The connection is stored and is based upon its port number. Meaning all send and recvs use the port number to differentiate the connects. This obviously has its flaws, but
provides a simple interface to the end user. This function has to be called first before recv or send.
\par
The recv\_tcp\_server will read data from the socket setup by setup\_tcp\_server. The port must match the port given setup. It will read and return the number of bytes into the vector.
This will also return the number of bytes read. Since this is a non-blocking function this can be zero.
\par
The send\_tcp\_server will write data to the socket setup by setup\_tcp\_server. The port must match the port given in setup. It will write and return the number of bytes written to the socket.
This will also return the number of bytes written. This is non-blocking and can write zero bytes if it is unable to.

Please see \ref{Code Documentation} for more information per target.

\section{Building}

\par
The all VPI TCP Server source files are written in C to target the VPI API from Verilog 2001. They should simulate in any modern simulation tool that has VPI support.
The library comes as a fusesoc packaged core and can be included in any other testbench. Be sure to make sure you have meet the dependencies listed in the previous section.

\subsection{fusesoc}
\par
Fusesoc is a system for building FPGA software without relying on the internal project management of the tool. Avoiding vendor lock in to Vivado or Quartus.
These cores, when included in a project, can be easily integrated and targets created based upon the end developer needs. The core by itself is not a part of
a system and should be integrated into a fusesoc based system. Simulations are setup to use fusesoc and are a part of its targets.

\subsection{Source Files}

\subsubsection{fusesoc\_info File List}
\begin{itemize}
\item common\_src
	\begin{itemize}
	\item {'src/tcp\_server.c': {'file\_type': 'cSource'}}
	\item {'src/tcp\_server.h': {'file\_type': 'cSource', 'is\_include\_file': True}}
	\item {'src/messages.h': {'file\_type': 'cSource', 'is\_include\_file': True}}
	\end{itemize}
\item vpi\_src
	\begin{itemize}
	\item {'src/vpi\_messages.c': {'file\_type': 'cSource'}}
	\item {'src/vpi\_tcp\_server.c': {'file\_type': 'cSource'}}
	\item {'src/vpi\_send\_tcp\_server.c': {'file\_type': 'cSource'}}
	\item {'src/vpi\_recv\_tcp\_server.c': {'file\_type': 'cSource'}}
	\item {'src/vpi\_tcp\_server.h': {'file\_type': 'cSource', 'is\_include\_file': True}}
	\item {'src/vpi\_send\_tcp\_server.h': {'file\_type': 'cSource', 'is\_include\_file': True}}
	\item {'src/vpi\_recv\_tcp\_server.h': {'file\_type': 'cSource', 'is\_include\_file': True}}
	\item {'src/vpi\_tcp\_server.sft': {'file\_type': 'user'}}
	\end{itemize}
\item verilator
	\begin{itemize}
	\item {'verilator\_messages.cpp': {'file\_type': 'cppSource'}}
	\end{itemize}
\item lib
	\begin{itemize}
	\item {'lib\_ringbuffer/build/libringBuffer.a': {'file\_type': 'user', 'copyto': '.'}}
	\end{itemize}
\item header
	\begin{itemize}
	\item {'lib\_ringbuffer/ringBuffer.h': {'file\_type': 'cSource', 'is\_include\_file': True}}
	\end{itemize}
\item tb
	\begin{itemize}
	\item {'tb/tb\_vpi.v': {'file\_type': 'verilogSource'}}
	\end{itemize}
\end{itemize}


\subsection{Targets} \label{targets}

\subsubsection{fusesoc\_info Targets}
\begin{itemize}
\item default
	\begin{itemize}
	\item[$\space$] Info: Intergration default target for simulations.
	\end{itemize}
\item sim
	\begin{itemize}
	\item[$\space$] Info: Test VPI TCP server.
	\end{itemize}
\end{itemize}


\subsection{Directory Guide}

\par
Below highlights important folders from the root of the directory.

\begin{enumerate}
  \item \textbf{docs} Contains all documentation related to this project.
    \begin{itemize}
      \item \textbf{manual} Contains user manual and github page that are generated from the latex sources.
    \end{itemize}
  \item \textbf{src} Contains source files for tcp server vpi interface.
  \item \textbf{tb} Contains test bench files.
\end{enumerate}

\newpage

\section{Simulation}
\par
A barebones test bench for iverilog is included in tb/tb\_vpi.v . This can be run from fusesoc with the following.
\begin{lstlisting}[language=bash]
$ fusesoc run --target=sim AFRL:vpi:tcp_server:1.0.0
\end{lstlisting}

\newpage

\section{Code Documentation} \label{Code Documentation}

\begin{itemize}
\item \textbf{TCP SERVER FILE SOURCE, DOXYGEN}
\end{itemize}
The next section documents the library.

